\documentclass[margin,line]{res}

% \usepackage[sfdefault]{roboto}
% \usepackage[T1]{fontenc}
%
\usepackage{hyperref}

\oddsidemargin -.5in
\evensidemargin -.5in
\textwidth=6.0in
\itemsep=0in
\parsep=0in
% if using pdflatex:
\setlength{\pdfpagewidth}{\paperwidth}
\setlength{\pdfpageheight}{\paperheight} 

\newenvironment{list1}{
  \begin{list}{\ding{113}}{%
      \setlength{\itemsep}{0in}
      \setlength{\parsep}{0in} \setlength{\parskip}{0in}
      \setlength{\topsep}{0in} \setlength{\partopsep}{0in} 
      \setlength{\leftmargin}{0.17in}}}{\end{list}}
\newenvironment{list2}{
  \begin{list}{$\bullet$}{%
      \setlength{\itemsep}{0in}
      \setlength{\parsep}{0in} \setlength{\parskip}{0in}
      \setlength{\topsep}{0in} \setlength{\partopsep}{0in} 
      \setlength{\leftmargin}{0.2in}}}{\end{list}}


\begin{document}

\name{Max Sutters \vspace*{.1in}}

\begin{resume}
\section{\sc Contact Information}
\vspace{.05in}
\begin{tabular}{@{}p{0.5in}p{2.5in}p{3in}}
{\it LinkedIn:} & https://www.linkedin.com/in/maxsutters      
							& {\it E-mail:}  msutters@cs.washington.edu \\   
% {\it GitHub:} & https://github.com/maximelearning
							& {\it Voice/text:}  (206) 321-0208         \\   
\end{tabular}

%\section{\sc Interests}
% Computer science education, inclusive hardware and software design, game development
% \vspace{-.3cm}

\section{\sc Education}
{\bf University of Washington}, Seattle, Washington USA\\
{\em Paul G. Allen School of Computer Science \& Engineering} 
%\vspace*{-.3cm}
\vspace*{.1cm} 
\begin{list1}
\item[] B.S. Candidate, Computer Engineering (expected
  graduation date: June 2021)
\end{list1}

{\bf Seattle Central College}, Seattle, Washington USA %\hfill GPA: {\bf 3.93}
\\

\vspace*{-.3cm}
\begin{list1}
\item[] A.S., Computer Science \& Engineering, June, 2019
\end{list1}

\vspace{-.2cm}
\section{\sc Technical Skills} 
	{\em Languages}:  
	Java, C/C++, Python, Unix/Bash shell scripting, HDL, assembly
	\\
	{\em Tools}:  
	GNU Debugger (GDB), Vim, Git/GitLab, IDEA, KiCad, \LaTeX, Mathematica 
	\\
	{\em Algorithm projects}: 
	Spam filter using machine learning (Naive Bayes), KD-tree
	nearest neighbor finder, content-aware image resizing with A* graph search
	\\
	{\em Operating Systems}:  
	Unix/Linux (CentOS, Ubuntu, WSL), Windows
	\\
	{\em Hardware}:  PCB design, 3D printing, flashing of Arduino/STM32 chips, SMD soldering
	\\

%%%%%%%%%%%%%%%%%%%%%%%%%%%%%%%%%%%%%%%%%%%%%%%%%%%%%
%\section{\sc Honors and Awards} % category below
%\vspace*{-2.5mm} % award below (repeat for more awards)

\vspace{-.3cm}
\section{\sc Experience}

% % \vspace{-.3cm}
% {\em Peer Advisor} \hfill {\bf April, 2020 - June 2021}\\
% Advising on academic questions within the CSE department at UW.

{\bf Seattle Central College}, Seattle, Washington USA

\vspace{-.3cm}
{\em Teaching Assistant} \hfill {\bf September, 2018  - March, 2019}\\
Duties included office hours, technical support, and management of cloud-based messaging forum. 
  Driver of Slack use in computer science classes at Seattle Central College.
\begin{list2}
\item CSC 110 Intro to Computer Programming with Clarke Wellman
\item CSC 142 Computer Programming I with François Lepeintre
\end{list2}     

{\bf SACNAS Chapter, Seattle Central College}, Seattle, Washington USA

\vspace{-.3cm}
{\em Chapter Secretary} \hfill {\bf May, 2019 - August, 2019}\\
Leading member of Society for the Advancement of Chicanos/Hispanics and Native Americans in Science.
  Organized meetings and researched chapter project proposals, synthesized team documentation, and 
  corresponded with chapter leadership and members. Volunteered for events including UndocuSTEM Conference.

\section{\sc Programming Projects}
{\bf malloc}, CSE 351

\vspace{-.3cm}
Implemented memory allocation in C.

{\bf HuskyMaps}, CSE 332

\vspace{-.3cm}
Used Java to implement a local navigation web application. Programmed a rasterization system for rendering tiles when zooming in and out of the HuskyMaps interface, and an A* graph-based text search for locations on the map. Hosted locally and on Heroku.

{\bf Tetris}, CSC 143

\vspace{-.3cm}
Practiced advanced object-oriented programming (OOP) by implementing Tetris in Java. Reinforced understanding of composition, inheritance, and model-view-controller (MVC) architecture. Applied unit testing, version control through Git, and pair programming.

{\bf Choose-Your-Own-Adventure Game Engine}, CSC 143

\vspace{-.3cm}
{\em (Credit to Stanford, Will Crowther)} Used Java to program a text-based RPG engine. Implemented a text file scanner to assign data to generic structures in the game. Developed object-oriented logic to allow players to travel between rooms, interact with objects, and solve puzzles.

\end{resume}
\end{document}




